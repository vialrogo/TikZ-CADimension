\usepackage{tikz}
\usepackage{xparse}
\usetikzlibrary{calc,arrows.meta}

% Default styles definition
\tikzset{% 
    prim-Dim/.style={%
      >={Latex[length=3mm, width=1.5mm]}, % Arrow head proprieties
      very thin,                          % Line style
      font=\footnotesize,                 % Dimension line text size
    },
    prim-DimNode/.style={prim-Dim,
      fill=white,                         % Fill with color background, by default white
      inner sep=0pt,                      % Space to the line 
      outer sep=0pt,                      % Space to the line 
    },
    Dimension/.style={},                  % Empty style for user use
    DimNode/.style={},                    % Empty style for user use
}

\makeatletter
\NewDocumentCommand{\Cote}{%
%-------------------------------------------------------------------------------------------%
%                                 Input Parameters                                          %
%-------------------------------------------------------------------------------------------%
% Type? | brackets | Default |                  Mean
    s                        % Is the star. With star the arrow is external.
    D       <>        {1pt}  % Offset of limit lines (deprecated!)
    O                 {7mm}  % Offset of dimension line
    m                        % First point
    m                        % Second point
    m                        % Label
    D       <>         {o}   % h: Horizontal, v: Vertical, r: Radius, a: Angle, o: Oblique
    O                  {}    % Parameters for include in tikzset
%-------------------------------------------------------------------------------------------%
}{%

{%%%%%%%%%%%%%%%%%%%%%%%%%%%%%%%%%  Start the body of Command  %%%%%%%%%%%%%%%%%%%%%%%%%%%%%%
  
  % User defined styles
  \tikzset{#8}

  % Total styles
  \tikzset{%
    Dim/.style  ={prim-Dim,Dimension},
    Node/.style ={prim-DimNode,DimNode},
  }

  % Coordinates calculation
  \coordinate (@1) at #4 ;
  \coordinate (@2) at #5 ;

  \if #7v % Vertical dimension
      \coordinate (@0) at ($($#4!.5!#5$) + (#3,0)$) ; 
      \coordinate (@4) at (@0|-@1) ;
      \coordinate (@5) at (@0|-@2) ;
  \else
  \if #7h % Horizontal dimension
      \coordinate (@0) at ($($#4!.5!#5$) + (0,#3)$) ; 
      \coordinate (@4) at (@0-|@1) ;
      \coordinate (@5) at (@0-|@2) ;
  \else 
  \if #7r % Radial dimension
      \coordinate (@4) at (@1) ;
      \coordinate (@5) at (@2) ;
  \else % cotation encoche
  \ifnum\pdfstrcmp{\unexpanded\expandafter{\@car#7\@nil}}{(}=\z@
  % \if #7a % Angular dimension
      \coordinate (@5) at ($#7!#3!#5$) ;
      \coordinate (@4) at ($#7!#3!#4$) ;
  \else % cotation oblique    
      \coordinate (@5) at ($#5!#3!90:#4$) ;
      \coordinate (@4) at ($#4!#3!-90:#5$) ;
  \fi\fi\fi\fi

  % Put the limit lines, only if isn't radius
  \if #7r
    % Center circle and center lines?
  \else
    \draw[Dim, shorten >= 1mm, shorten <= -3mm] (@4) -- #4 ;
    \draw[Dim, shorten >= 1mm, shorten <= -3mm] (@5) -- #5 ;
  \fi

  \IfBooleanTF #1 {% Con asterisco
    \draw[Dim,-] (@4) -- (@5) node[Node] at ($ 0.5*(@4) + 0.5*(@5) $) {#6\strut};
    \draw[Dim,<-] (@4) -- ($(@4)!-6pt!(@5)$) ;   
    \draw[Dim,<-] (@5) -- ($(@5)!-6pt!(@4)$) ;   
  }{%Sin asterisco
  \ifnum\pdfstrcmp{\unexpanded\expandafter{\@car#7\@nil}}{(}=\z@
    \draw[Dim, <-> ] (@5) to[bend right] node[Node] {#6\strut} (@4) ;
  \else
    \if #7r
      \draw[Dim, ->] (@4) -- (@5) node[Node] at ($ 0.5*(@4) + 0.5*(@5) $) {#6\strut};
    \else
      \draw[Dim, <->] (@4) -- (@5) node[Node] at ($ 0.5*(@4) + 0.5*(@5) $) {#6\strut};
    \fi
  \fi
  }}
}
\makeatother

