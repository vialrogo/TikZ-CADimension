\usepackage{tikz}
\usepackage{xparse}
\usetikzlibrary{calc,arrows.meta}

% Default styles definition
% This has to be here, because is executed when the library is load, before command invocation.
% This is very important for avoid the re-definition of user styles
\tikzset{% 
    prim-Dim/.style={%
      >={Latex[length=3mm, width=1.5mm]}, % Arrow head proprieties
      very thin,                          % Line style
      font=\footnotesize,                 % Dimension line text size
    },
    prim-DimNode/.style={prim-Dim,
      fill=white,                         % Fill with color background, by default white
      inner sep=1pt,                      % Space to the line 
      outer sep=1pt,                      % Space to the line 
    },
    Dimension/.style={},                  % Empty style for user use
    DimNode/.style={},                    % Empty style for user use
}

\makeatletter
\NewDocumentCommand{\CADim}{%
%-------------------------------------------------------------------------------------------%
%                                 Input Parameters                                          %
%-------------------------------------------------------------------------------------------%
% Type? | brackets | Default |                  Mean
    s                        % #1 Is the star. With star the arrow is external.
    D       <>         {o}   % #2 Label position (it's a point)
    O                 {10mm} % #3 Offset of dimension line
    m                        % #4 First point
    m                        % #5 Second point
    D       <>         {o}   % #6 Third point. It's useful in angle dimension
    m                        % #7 Label
    D       <>         {o}   % #8 h: Horizontal, v: Vertical, r: Radius, a: Angle, o: Oblique
    O                  {}    % #9 Parameters for include in tikzset
%-------------------------------------------------------------------------------------------%
}{%

{%%%%%%%%%%%%%%%%%%%%%%%%%%%%%%%%%  Start the body of Command  %%%%%%%%%%%%%%%%%%%%%%%%%%%%%%
  
  % User defined styles
  \tikzset{#9}

  % Total styles
  \tikzset{%
    Dim/.style  ={prim-Dim,Dimension},
    Node/.style ={prim-DimNode,DimNode},
  }

  %---------------------------------- Coordinates calculation -------------------------------
  % (@0) -> Is the label position
  % (@1) -> Is the first limit line point
  % (@2) -> Is the second limit line point
  % (@3) -> Is the third limit line point (only for angular dimension)
  % (@4) -> Is the first dimension line point 
  % (@5) -> Is the second dimension line point 

  \coordinate (@1) at #4;
  \coordinate (@2) at #5;

  \if #8v % Vertical dimension
    \coordinate (@0) at ($($#4!.5!#5$) + (#3,0)$); 
    \coordinate (@4) at (@0|-@1);
    \coordinate (@5) at (@0|-@2);

  \else\if #8h % Horizontal dimension
    \coordinate (@0) at ($($#4!.5!#5$) + (0,#3)$); 
    \coordinate (@4) at (@0-|@1);
    \coordinate (@5) at (@0-|@2);

  \else\if #8r % Radial dimension
      \coordinate (@4) at (@1);
      \coordinate (@5) at (@2);

  \else\if #8a % Angular dimension
      \coordinate (@3) at ($#6$);
      \coordinate (@4) at ($#6!#3!#4$);
      \coordinate (@5) at ($#6!#3!#5$);

  \else\if #8o % Oblique dimension   
      \coordinate (@4) at ($#4!#3!-90:#5$);
      \coordinate (@5) at ($#5!#3!90:#4$);

  \fi\fi\fi\fi\fi

  %-------------------------------------- Put limit lines -----------------------------------

  \if #8r % If is radius dimension
    \draw[Dim, fill] (@1) circle (0.3mm);
    \draw[Dim] ($(@1) + (0,1.5mm)$) -- ($(@1) + (0,-1.5mm)$);
    \draw[Dim] ($(@1) + (1.5mm,0)$) -- ($(@1) + (-1.5mm,0)$);

  \else % If isn't radius dimension
    \draw[Dim, shorten >= 1mm, shorten <= -3mm] (@4) -- (@1);
    \draw[Dim, shorten >= 1mm, shorten <= -3mm] (@5) -- (@2);

  \fi

  %-----------------------------  Calculate the label position  ------------------------------

  \if #2o % If point isn't specified
    \if #8a % Angular dimension
      \coordinate (@0) at ($(@3)!#3!($0.5*(@4) + 0.5*(@5)$)$);

    \else\if #8r % Radial dimension
      \IfBooleanTF #1 {% With star
        \coordinate (@0) at ($(@5)!-10mm!(@4)$);
      }{% Without star
        \coordinate (@0) at ($0.5*(@4) + 0.5*(@5)$);
      }

    \else % Others dimensions
      \coordinate (@0) at ($0.5*(@4) + 0.5*(@5)$);

    \fi\fi

  \else % The point is specified
    \coordinate (@0) at #2;

  \fi

  %----------------------------------- Put dimension lines -----------------------------------

  \IfBooleanTF #1 {% With star
    
    \if #8a % Angular dimension
      \pgfmathanglebetweenpoints{\pgfpointanchor{@3}{center}}{\pgfpointanchor{@4}{center}}
      \let\AngleA\pgfmathresult
      \pgfmathanglebetweenpoints{\pgfpointanchor{@3}{center}}{\pgfpointanchor{@5}{center}}
      \let\AngleB\pgfmathresult
      \def\Delta{300mm/#3} % This fix always 5,2mm for the line
      \draw[Dim,<-] (@4) arc (\AngleA:\AngleA+\Delta:#3);
      \draw[Dim,<-] (@5) arc (\AngleB:\AngleB-\Delta:#3);
      \node[Dim,Node] at (@0) {#7\strut};

    \else\if #8r % Radial dimension
      \coordinate (LAB) ato ($(@0) + (#3,0)$);
      \draw[Dim, <-] (@5) -- (@0) -- (LAB) node[Dim,Node] at (LAB) {#7\strut};

    \else % Others dimensions
      \node[Dim,Node] at (@0) {#7\strut};
      \draw[Dim,<-] (@4) -- ($(@4)!-5mm!(@5)$);   
      \draw[Dim,<-] (@5) -- ($(@5)!-5mm!(@4)$);   

    \fi\fi
  }{% Without star

    \if #8a % Angular dimension
      \pgfmathanglebetweenpoints{\pgfpointanchor{@3}{center}}{\pgfpointanchor{@4}{center}}
      \let\AngleA\pgfmathresult
      \pgfmathanglebetweenpoints{\pgfpointanchor{@3}{center}}{\pgfpointanchor{@5}{center}}
      \let\AngleB\pgfmathresult
      \draw[Dim, <->] (@4) to[out=\AngleA-90, in=\AngleB+90]  node[Dim,Node] {#7\strut} (@5);

    \else\if #8r % Radial dimension
      \draw[Dim, ->] (@4) -- (@5) node[Dim,Node] at (@0) {#7\strut};
    
    \else % Others dimensions
      \draw[Dim, <->] (@4) -- (@5) node[Dim,Node] at (@0) {#7\strut};
    
    \fi\fi
  }}
}
\makeatother

